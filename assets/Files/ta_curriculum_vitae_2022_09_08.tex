% Created 2022-09-08 Thu 09:51
% Intended LaTeX compiler: pdflatex
\documentclass[11pt]{article}
\usepackage[utf8]{inputenc}
\usepackage[T1]{fontenc}
\usepackage{graphicx}
\usepackage{grffile}
\usepackage{longtable}
\usepackage{wrapfig}
\usepackage{rotating}
\usepackage[normalem]{ulem}
\usepackage{amsmath}
\usepackage{textcomp}
\usepackage{amssymb}
\usepackage{capt-of}
\usepackage{hyperref}
\author{Travis Alongi}
\date{\today}
\title{Curriculum Vitae}
\hypersetup{
 pdfauthor={Travis Alongi},
 pdftitle={Curriculum Vitae},
 pdfkeywords={},
 pdfsubject={},
 pdfcreator={Emacs 28.1 (Org mode 9.5)}, 
 pdflang={English}}
\begin{document}

\maketitle
Travis Alongi
UC Santa Cruz Seismological Laboratory
1156 High st., Earth \& Marine Sciences, Santa Cruz CA 9564

\section{Research Interests}
\label{sec:org96970a4}
Offshore faults, fault damage zones, subduction zone seismology, seismic and aseismic slip.

\section{Education}
\label{sec:orgfbf7b53}
\begin{itemize}
\item PhD, Earth and Planetary Science (Expected graduation summer 2023)
\begin{itemize}
\item University of California, Santa Cruz
\end{itemize}

\item UC Extension program - Earth Science Coursework
\begin{itemize}
\item University of California, Santa Cruz [GPA 4.0]
\end{itemize}

\item Geology, Science, Mathematics
\begin{itemize}
\item Cabrillo College, Aptos, CA [GPA 3.7]
\end{itemize}

\item Bachelor of Science, Business (Dec. 2007)
\begin{itemize}
\item San Jose State University, CA [GPA 3.52] Dean’s Scholar
\end{itemize}
\end{itemize}

\section{Publications}
\label{sec:orgbeca5d2}
\textbf{Alongi, T.}, Brodsky, E.E., Kluesner, J.W., Brothers, D.S., 2022, Using Active Source Seismology to Image the Palos Verdes Fault Damage Zone as a Function of Distance, Depth, and Geology, Earth and Planetary Science Letters [resubmitted minor revisions Sept. 2022]

\textbf{Alongi, T.}, Balster-Gee, A.F., Kluesner, J.W., Sliter, R.W., 2022, Reprocessed multichannel seismic-reflection data acquired offshore Southern California during USGS field activity O-1-99-SC: U.S. Geological Survey data release, \url{https://doi.org/10.5066/P9GR0PWF}

\textbf{Alongi, T.}, Balster-Gee, A.F.,Kluesner, J.W., Sliter, R.W., 2022, Reprocessed multichannel seismic-reflection data collected offshore central and Southern California during USGS field activity L-4-90-SC: U.S. Geological Survey data release,\url{https://doi.org/10.5066/P9FOES4K}

\textbf{Alongi, T.}, Schwartz, S. Y., Shaddox, H. R., \& Small, D. T. (2021). Probing the Southern Cascadia Plate Interface with the Dense Amphibious Cascadia Initiative Seismic Array. Journal of Geophysical Research: Solid Earth, 126, e2021JB022180. \url{https://doi.org/10.1029/2021JB022180}

\section{Research Position}
\label{sec:orgfd707ef}
\begin{itemize}
\item Using 3D Seismic Data to Study the Offshore Damage Zones, Kinematics and Earthquakes in the Palos Verdes Fault Region
\begin{itemize}
\item UC Santa Cruz and USGS Pacific Marine Coastal Science Center (8/2019 - Current)
\end{itemize}

\item Exploring seismicity of Southernmost Cascadia Subduction Zone Using Dense Seismic Network Including Ocean Bottom Seismometers
\begin{itemize}
\item Institute for Geophysics and Planetary Physics, UC Santa Cruz, CA (1/2018 - 3/2021)
\end{itemize}

\item Refining Slab Geometry \& Geodynamic Models of the Tonga Subduction Zone
\begin{itemize}
\item Scripps Institute of Oceanography, La Jolla, CA (7/2017-10/2017)
\end{itemize}
\end{itemize}

\section{Presentations}
\label{sec:org8e01b9a}
\begin{itemize}
\item Using Active Source Seismology to Image a Fault Damage Zone as a Function of Depth, Distance, and Geology (poster)
\begin{itemize}
\item Southern California Earthquake Center Annual Meeting
\item 9/2022 Palm Springs, CA
\end{itemize}

\item Using Active Source Seismology to Image a Fault Damage Zone as a Function of Depth, Distance, and Geology (poster)
\begin{itemize}
\item Gordon Research Conference: Rock Deformation
\item 8/2022 Lewiston, ME
\end{itemize}

\item Using Active Source Seismology to Image a Fault Damage Zone as a Function of Depth, Distance, and Geology (talk)
\begin{itemize}
\item Seismological Society of American Annual Meeting
\item 4/2022 Bellevue, WA
\end{itemize}

\item Using Active Source Seismology to Image a Strike-Slip Fault Damage Zone as a Function of Depth, Distance, and Geology (talk)
\begin{itemize}
\item American Geophysical Union Conference
\item 12/2021 New Orleans, LA
\end{itemize}

\item Using Active Source Seismology to Image a Strike-Slip Fault Damage Zone as a Function of Depth, Distance, and Geology (talk)
\begin{itemize}
\item 3rd Cargese Earthquakes School
\item 10/2021 Corsica, France
\end{itemize}

\item Using Active Source Seismology to Image a Strike-Slip Fault Damage Zone as a Function of Depth, Distance, and Geology (talk)
\begin{itemize}
\item Southern California Earthquake Center Annual Meeting
\item 9/2021 Virtual Meeting
\end{itemize}

\item Probing the Southern Cascadia Plate Interface with a Dense Amphibious Cascadia Initiative Seismic Array (talk)
\begin{itemize}
\item GAGE-SAGE Community Science Workshop
\item 8/2021 Virtual Meeting
\end{itemize}

\item Probing the Southern Cascadia Plate Interface with a Dense Amphibious Cascadia Initiative Seismic Array (talk)
\begin{itemize}
\item Northern California Earthquake Hazards Workshop
\item 2/2021 Virtual Meeting
\end{itemize}

\item Probing the Southern Cascadia Plate Interface with a Dense Amphibious Cascadia Initiative Seismic Array (talk)
\begin{itemize}
\item American Geophysical Union Conference
\item 12/2020 Virtual Meeting
\end{itemize}

\item Fault Damage Zones in 3D with Active-Source Seismic Data (poster)
\begin{itemize}
\item American Geophysical Union Conference
\item 12/2019 San Francisco, CA
\end{itemize}

\item Fault Damage Zones in 3D with Active-Source Seismic Data (poster)
\begin{itemize}
\item Southern California Earthquake Center Annual Meeting
\item 9/2019 Palm Springs, CA
\end{itemize}

\item Using the Cascadia Initiative to Investigate Seismicity and Possible Shallow Slow Slip Along the Southernmost Section of the Cascadia Subduction Zone. (poster)
\begin{itemize}
\item American Geophysical Union Conference
\item 12/2018 Washington D.C.
\end{itemize}

\item Refining the Tonga Slab Geometry Using Slab Phases of Seismic Waves
\begin{itemize}
\item American Geophysical Union Conference (poster)
\item 12/2017 New Orleans, LA
\end{itemize}
\end{itemize}

\section{Honors and Awards}
\label{sec:org697a04f}
\begin{itemize}
\item 2021 Zhen and Ren Wu Memorial Fund
\item 2020 Eli Silver EPS Opportunities Fund
\item 2017 IRIS Summer Internship
\item 2016 Henry A Martin Scholarship
\end{itemize}

\section{Conference convenership}
\label{sec:org077ba9f}
\begin{itemize}
\item 2022 Seismological Society of America Meeting, Convener
\emph{Fault Damage Zones: What We Know and Do Not (1 \& 2)}
\end{itemize}

\section{Field Experience}
\label{sec:org5b90801}
\begin{itemize}
\item 2021 RV Sproul
\begin{itemize}
\item Data collected: sparker MCS, chirp
\item Location: Offshore southern California, San Pedro shelf and slope
\end{itemize}
\item 2019 RV Bold Horizon
\begin{itemize}
\item Data collected: sparker MCS, chirp, piston Core
\item Location: Offshore northern California \& Oregon
\end{itemize}
\item 2018 Blue Mountain Geothermal
\begin{itemize}
\item Data collected: well water level and temperature
\item Location: Winnemucca, NV
\end{itemize}
\item 2017 IRIS pascal
\begin{itemize}
\item Data collected: passive seismometer installation
\item Location: Socorro, NM
\end{itemize}
\end{itemize}

\section{Teaching Experience}
\label{sec:orgf9e2bcb}
\begin{itemize}
\item Teaching Assistant, Geophysical Data Science (9/2021 - 12/2021)
\begin{itemize}
\item University of California, Santa Cruz, CA
\end{itemize}
\item Teaching Assistant, Environmental Geology (3/2020 - 7/2020)
\begin{itemize}
\item University of California, Santa Cruz, CA
\end{itemize}
\item Teaching assistant, Geology of National Parks (4/2019 - 7/2019)
\begin{itemize}
\item University of California, Santa Cruz, CA
\end{itemize}
\item Teaching assistant, Environmental Geology (4/2020 - 7/2020)
\begin{itemize}
\item University of California, Santa Cruz, CA
\end{itemize}
\item Student assistant, California Historical Geology (1/2016 – 7/2016)
\begin{itemize}
\item Cabrillo College, Aptos, CA
\end{itemize}
\end{itemize}

\section{Relevant Coursework}
\label{sec:orgd4aded4}
Earthquake Physics, Crustal Deformation, Order of Magnitude Estimation, The Dynamic Earth, Practical Geophysics, Seismotectonics, Machine Learning for Geophysicists, Topics in Geophysics, Scientific Computing, Foundations in Applied Mathematics, Structural Geology, Data Analysis in Earth Science, Foundations in Earth Science.
\end{document}
